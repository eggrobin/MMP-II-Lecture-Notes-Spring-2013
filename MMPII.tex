%%%%%%%%%%%%%%%%%%%%%%%%%%%%%%%%%%%%%%%%%%%%%%%%%%%%
%%%%%%%%%%  Please follow ISO standards.  %%%%%%%%%%
%%%%%%%%%%%%%%%%%%%%%%%%%%%%%%%%%%%%%%%%%%%%%%%%%%%%
%  A copy of ISO 80000-2:2009 can be found at      %
%  <http://goo.gl/fVoiF>. Keep other applicable    %
%  standards in   mind, e.g. ISO 8601:2004, etc.   %
%%%%%%%%%%%%%%%%%%%%%%%%%%%%%%%%%%%%%%%%%%%%%%%%%%%%
% build options (usage:                            %
% pdflatex "<options>%%%%%%%%%%%%%%%%%%%%%%%%%%%%%%%%%%%%%%%%%%%%%%%%%%%%
%%%%%%%%%%  Please follow ISO standards.  %%%%%%%%%%
%%%%%%%%%%%%%%%%%%%%%%%%%%%%%%%%%%%%%%%%%%%%%%%%%%%%
%  A copy of ISO 80000-2:2009 can be found at      %
%  <http://goo.gl/fVoiF> Keep other applicable     %
%  standards in   mind, e.g. ISO 8601:2004, etc.   %
%%%%%%%%%%%%%%%%%%%%%%%%%%%%%%%%%%%%%%%%%%%%%%%%%%%%
\documentclass[10pt]{article}

% Set ComputerModern to true to use fonts from the Computer Modern family,
% Utopia to true to use the Utopia family. Set none to true for compilation 
% errors, or both for havoc.
\newif\ifComputerModern
\newif\ifUtopia
\Utopiatrue
%\ComputerModerntrue

\usepackage{amsmath}
\usepackage{amsthm}
\usepackage{mathtools}
\usepackage[top=1.5cm, bottom=1.5cm, left=2.5cm, right=1.5cm]{geometry}

\ifUtopia
  \usepackage[utopia]{mathdesign}
  \usepackage[OMLmathrm,OMLmathbf]{isomath}
\fi
\ifComputerModern
  \usepackage{amsfonts}
  \usepackage{upgreek}
\fi

\usepackage{isomath}

\newtheorem*{lemma}{Lemma}
\newtheorem*{proposition}{Proposition}

% Please declare separate commands for distinct objects, e.g., the scalar
% product on R^n and the one on S(R^n) should not bear the same name, as
% we might decide to write them differently at some point.

% Always use the delimiters with a '*', as in \abs*, \of*, as it makes them
% scale properly
\DeclarePairedDelimiter\norm{\Vert}{\Vert}
\DeclarePairedDelimiter\abs{\lvert}{\rvert}
\DeclarePairedDelimiterX\comm[2]{[}{]}{#1, #2}
%\DeclarePairedDelimiterX\scal[2]{\langle}{\rangle}{#1\delimsize\vert #2}
\DeclarePairedDelimiterX\scal[2]{.}{.}{#1\cdot #2}

% Parentheses. Do not use standard parentheses '(blah)' in math
% mode, as these do not scale. use '\of*{blah}' instead, or '\of* b' instead of '(b)'
\DeclarePairedDelimiter\of{\lparen}{\rparen}

\newcommand{\R}{\mathbb{R}}
\newcommand{\Rn}{{\R^n}}
\newcommand{\Schwartz}{{\mathcal{S}\of*{\R^n}}}
\DeclareMathOperator{\diffd}{d}
\DeclareMathOperator{\FT}{\mathcal{F}}
\DeclareMathOperator{\End}{End}

%%% begin Voodoo

% This conjures some dark TeX magic to create a \widecheck.

\makeatletter
\newdimen\rh@wd
\newdimen\rh@hta
\newdimen\rh@htb
\newbox\rh@box
\def\rh@measure#1{\setbox\rh@box=\hbox{$#1$}\rh@wd=\wd\rh@box \rh@hta=\ht\rh@box}

\def\widecheck#1{\rh@measure{#1}%
  \setbox\rh@box=\hbox{$\widehat{\vrule height \rh@hta width\z@ \kern\rh@wd}$}%
  \rh@htb=\ht\rh@box \advance\rh@htb\rh@hta \advance\rh@htb\p@
  \ooalign{$\vrule height \ht\rh@box width\z@ #1$\cr
           \raise\rh@htb\hbox{\scalebox{1}[-1]{\box\rh@box}}\cr}}
\makeatother
    
%%% end Voodoo

% For faster typing, and disambiguation between \gp (a variable called pi) and \pi
% The smallest positive half-period of the sine.
\newcommand\gl\lambda
\newcommand\gj\varphi
\newcommand\gy\psi
\renewcommand{\ge}{\epsilon}
\newcommand{\gd}{\delta}
\let\gp\pi

% Vectors are typeset in bold italic (isomath \vectorsym) as per ISO 80000-2:2009
\newcommand{\bx}{{\vectorsym x}}
\newcommand{\by}{{\vectorsym y}}
\newcommand{\bk}{{\vectorsym k}}
\newcommand{\bga}{{\vectorsym \alpha}}
\newcommand{\bgb}{{\vectorsym \beta}}

% Several shorthands to reduce the number of '{', '}', '_', and other tiresome characters.
\newcommand\ft\widehat
\newcommand\rft\widecheck
\newcommand\Int[1]{\int\limits_#1}
\newcommand{\Sum}{\sum\limits}
\newcommand{\adj}{^\dagger}

\newcommand\fstderiv[2][]{\frac{\diffd #1}{\diffd #2}}
\newcommand\nthderiv[3][]{\frac{\diffd^#3 #1}{{\diffd #2}^#3}}
\newcommand\fstpartial[2][]{\frac{\partial #1}{\partial #2}}
\newcommand\nthpartial[3][]{\frac{\partial^#3 #1}{{\partial #2}^#3}}
\newcommand{\pd}[1]{{\partial_#1}}

% For comments in align environments
\newcommand\commentbox[1]{\parbox{7.5cm}{#1}}

% Constants are typeset in roman as per ISO 80000-2:2009
\ifUtopia \renewcommand{\pi}{{\mathrm \gp}} \fi
\ifComputerModern \renewcommand{\pi}{\uppi} \fi
\newcommand\I{\mathrm{i}}
\newcommand\E{\mathrm{e}}

% As it turns out, you write this at every single line.
\newcommand{\sqftnrm}{\frac{1}{\of*{2\pi}^n} }
\newcommand{\ftnrm}{\frac{1}{\of*{2\pi}^\frac{n}{2}} }

\usepackage{datetime} 
\renewcommand{\dateseparator}{-}
\yyyymmdddate 

\title{Notes from the Methods of Mathematical Physics II courses of 2013-02-19{\slash}05-30 by Prof.~Dr.~Eugene~Trubowitz}
\author{Robin~Leroy and David~Nadlinger}

\begin{document}
  \maketitle
  \section{General properties of the Fourier transform}
  \noindent \emph{Here begins the lecture of 2013-03-06.}
  \begin{lemma}[``Your life in Fourier land depends on it.'']
    \begin{align}
      \label{FTLinearity}
      \begin{split}      
      \ft{\gj + \gy} &= \ft{\gj} + \ft{\gy},\\
      \ft{\gl \gj} &= \gl \ft{\gj} 
      \end{split}
      \\
      \label{FTUpperBound}
      \abs*{\ft{\gj}(\bk)} &\leq \frac{1}{\of*{2\pi}^\frac{n}{2}}\norm*{\gj}_1<\infty, &&
      \norm*{\gj}_1 \coloneqq \Int\Rn \abs*{\gj(\bx)} \diffd \bx
      \\
      \label{FTScaling}
      \ft{\gj\of*{\gl\cdot}}\of*\bk &= \frac{1}{\abs*{\gl}^n}\ft{\gj}\of*{\frac\bk\gl}, &&
      \gl\neq 0
      \\
      \label{FTTranslation}
      \ft{T_\by\gj}\of*\bx &= \E^{\I\scal*\bk\by}\ft\gj\of*\bk, &&
      \of*{T_\by\gj}\of*\bx \coloneqq \gj\of*{\bx+\by}
      \\
      \label{FTDerivative}
      \begin{split}
      \fstpartial[\gj]{x_j}\of*\bk &= \I k_j \ft\gj\of*\bk \\ 
      \ft{x_j \gj}\of*\bk &= \I \fstpartial[\ft\gj]{k_j} \of*\bk
      \end{split}
      \\
      \label{FTUnitary}
      \Int\Rn\ft\gj\of*\by \gy\of*\by \diffd\by &= \Int\Rn \gj\of*\by \ft\gy\of*\by \diffd\by
    \end{align}
    \begin{proof}
      (\ref{FTUpperBound}):
      \begin{equation*}
       \abs*{\ft\gj\of*\bk} 
        = \abs*{\ftnrm\Int\Rn \gj\of*\bx \E^{-\I \scal*\bk\bx}\diffd\bx} 
        \leq \ftnrm\Int\Rn \abs*{ \gj\of*\bx 
        \E^{-\I \scal*\bk\bx}}\diffd\bx  = \frac{1}{\of*{2\pi}^\frac{n}{2}}
        \underbrace{\Int\Rn \abs*{\gj\of*\bx}}_{\norm{\gj}_1}
        \underbrace{\abs*{\E^{-\I\scal*\bk\bx}}}_{1}\diffd\bx \\
      \end{equation*}
      Proof of existence of the integral $\norm\gj_1$:
      \begin{align*}
        \Int\Rn\abs*{\gj\of*\bx}\diffd\bx 
        &= \Int\Rn\of*{1+\abs*{\bx}^2}^\frac{s}{2}\abs*{\gj\of*\bx}
        \frac{\diffd\bx}{\of*{1+\abs*{\bx}^2}^\frac{s}{2}}  \\
        &\leq \norm{\gj}_{s,0} \Int\Rn\frac{1}{\of*{1+\abs*{\bx}^2}^\frac{s}{2}}\diffd\bx\\
        &= \norm{\gj}_{s,0} \Int{0}^\infty\frac{r^{n-1}}{\of*{1+r^2}^\frac{s}{2}}\diffd r
        < \infty &&\text{for } s\geq n+1
      \end{align*}
      ``(\ref{FTLinearity}) I will not do!''
      Proof of (\ref{FTScaling}):
      \begin{align*}
        \ft{\gj\of*{\gl\cdot}} 
        &= \ftnrm\Int\Rn\gj\of*{\gl\bx}\E^{-\I\scal*\bk\bx} \diffd\bx \\
        &= \frac{1}{\of*{2\pi}^\frac{n}{2}}
          \Int\Rn\gj\of*\by\E^{-\I\scal*{\frac 1\gl\bk}{\by}} \diffd\of*{\frac{\by}{\gl}}
          && \text{where } \bx=\frac{1}{\gl}\by \\
        &= \frac{1}{\gl^n} \underbrace{
          \ftnrm\Int\Rn\gj\of*\by\E^{-\I\scal*{\frac\bk\gl}\by}
          \diffd\by}_{\ft\gj\of*{\frac\bk{\gl}}}
      \end{align*}
      Proof of the second part of (\ref{FTDerivative}):
      \begin{align*}
        \ft{x_j\gj}\of*\bx 
        &= \ftnrm\Int\Rn x_j\gj\of*{\bx}\E^{-\I\scal*\bk\bx}\diffd\bx \\
        &= \ftnrm
          \Int\Rn\gj\of*{\bx}\I\fstpartial{k_j}\E^{-\I\scal*\bk\bx} \diffd\bx 
          && \text{as }\fstpartial{k_j}\E^{-\I\scal*\bk\bx} =
            \fstpartial{k_j}\E^{-\I\Sum_{r=0}^n k_j x_j}=-\I x_j \E^{-\I\scal*\bk\bx}\\
        &= \I\fstpartial{k_j}\ftnrm
          \Int\Rn\gj\of*\bx\E^{-\I\scal*\bk\bx} \diffd\bx \\
        &= \I\fstpartial[\ft\gj]{k_j}\of*\bk
      \end{align*}
      
      \emph{The professor starts whistling some elevator music while waiting for a student to write down the proof before he can wipe the board. When the student is done, the professor notices that there still is some room left on the blackboard and starts writing the rest there, without erasing anything.}
      
      Proof of \ref{FTUnitary}:
      \begin{align*}
        \Int\Rn\ft\gj\of*\by \gy\of*\by \diffd\by
        &= \Int\Rn\frac{1}{\of*{2\pi}^\frac{n}{2}}\Int\Rn\gj\of*\bx
          \E^{-\I\scal*\by\bx}\diffd\bx\:\gy\of*\by\diffd\by \\
        &= \Int\Rn\gj\of*\bx\frac{1}{\of*{2\pi}^\frac{n}{2}}\Int\Rn
          \E^{-\I\scal\bx\by}\gy\of*\by\diffd\by\diffd\bx
          && \commentbox{As $\Int{\R^{2n}}\abs*{\gj\of*\bx
          \E^{-\I\scal\by\bx}\gy\of*\by}\diffd\bx\diffd\by<\infty$,
          we can use Fubini's theorem.} \\
        &=\Int\Rn\gj\of*\by \ft\gy\of*\by \diffd\by
      \end{align*}
      
      
    \end{proof}
  \end{lemma}
  
  \begin{proposition}
  The Fourier transform $\ft\quad$ is a continuous bijective linear map from the Schwartz space $\Schwartz$ to itself. $\rft{\ft{\gj}}=\gj$, where $\rft\gy\of*\bx\coloneqq\frac{1}{\of*{2\pi}^\frac{n}{2}}\Int\Rn\gy\of*\bk\E^{\I\scal*\bk\bx}\diffd\bx$, read ``unhat'', ``bird'', ``seagull'', or whatever you like.
  
    \begin{proof}
      \begin{align*}
        \gj\in\Schwartz&\Rightarrow\bx^\bga\pd\bx^\bgb\in\Schwartz \\
        \ft{\of*{\bx^\bga\pd\bx^\bgb\gj}}\of*\bk
        &=\I^{\abs*\bga}\pd\bk^\bga\ft{\of*{\pd\bx^\bgb\gj}}\of*\bk 
          && \text{from the second equality in (\ref{FTDerivative}), applied }\abs*\bga\text{ times.} \\
        &=\I^{\abs*\bga}\pd\bk^\bga\of*{\I^{\abs*\bgb}\bk^\bgb\ft\gj}\of*\bk \\
        &=\I^{\abs*\bga+\abs*\bgb}\pd\bk\of*{\bk^\bgb\ft\gj}\of*\bk
      \end{align*}
      ``I should have done it the other way around. [...] Let's try it the other way around.''
      We want:
      \begin{equation*}
        \forall\bga,\bgb,\sup_{\bk\in\R^n}\abs*{\bk^\bga\pd\bk^\bgb\ft\gj\of*\bk}<\infty
      \end{equation*}
      As that implies $\ft\gj\in\Schwartz$.
      \begin{align*}
        \ft{\of*{\pd\bx^\bga\of*{\bk^\bgb\gj}}}\of*\bk
        &=\I^{\abs*\bga+\abs*\bgb}\bk^\bga\pd\bk^\bgb\ft\gj\of*\bk
        && \commentbox{by applying the first part of (\ref{FTDerivative}) $\abs*\bga
        $ times and the second part $\abs*\bgb$ times.}\\
        \forall\bga,\bgb,
        \abs*{\bk\bga\pd\bk^\bgb\ft\gj\of*\bk}&\leq\ftnrm\norm*{\pd\bx\of*{\bx^\bgb\gj}}_1<\infty
        && \text{from (\ref{FTUpperBound}).}
      \end{align*}
      We now know $\ft\gj\in\Schwartz$.
      Let us prove $\rft{\ft\gj}=\gj$. ``When you do the wrong thing I'm gonna scream loudly.''
      \begin{align*}
        \rft{\ft\gj}\of*\bx&=\ftnrm\Int\Rn\ft\gj\of*\bk\E^{\I\scal*\bk\bx}\diffd\bk \\
        &=\sqftnrm\Int\Rn\Int\Rn\gj\of*\by\E^{-\I\scal*\bk\by}\diffd\by\E^{\I\scal*\bk\bx}\diffd\bk
      \end{align*}
      We don't interchange the integrals here because that would lead to ugly calculations. ``If pou paint the walls before you start building, it's not a good idea.''
      At this point, somebody suggests replacing $\sqftnrm\Int\Rn\E^{\I\scal*\bk{\bx-\by}}\diffd\bk$ by $\gd\of*{\bx-\by}$. ``It's plausible! --- WHY? [...] Ah I said it, so it's plausible.'' However, doing this would also lead to some nasty calculations. ``How are we going to do it so fast that you don't get bored, and yet in enough detail that he's convinced? Be sneaky.''
      \begin{align*}
        \rft{\ft{\gj}}
        &=\ftnrm\Int\Rn\ft\gj\of*\bk\E^{\I\scal*\bk\bx}
        \underbrace{\lim_{\ge\downarrow 0}\E^{-\ge\frac{\abs*\bk^2}{2}}}_{\mathclap{\substack{1
        \text{ written in}\\\text{some other way}}}}\diffd\bk\\
        &=\lim_{\ge\downarrow 0}\ftnrm\Int\Rn\ft\gj\of*\bk\E^{\I\scal*\bk\bx}
        \E^{-\ge\frac{\abs*\bk^2}{2}}&&\commentbox{``Don't worry.'' This is actually a one-liner 
        using Lebesgue's dominated convergence theorem.}
      \end{align*}
      \emph{The 10-minute break ends with the loud noise of a metallic pointing stick hitting the desk.}
      \begin{align*}
        \ftnrm\Int\Rn\ft\gj\of*\bk\E^{\I\scal*\bk\bx}\E^{-\ge\frac{\abs*\bk^2}{2}}\diffd\bk
        &=\ftnrm\Int\Rn\ftnrm\Int\Rn\gj\of*\by\E^{-\I\scal*\by\bk}\diffd\by\:
        \E^{\I\scal*\bk\bx}\E^{-\ge\frac{\abs*\bk^2}{2}}\diffd\bk\\
        &=\ftnrm\Int\Rn\gj\of*\by\ftnrm\Int\Rn\E^{\I\scal*\bk{\bx-\by}}
        \E^{-\ge\frac{\abs*\bk^2}{2}}\diffd\bk\diffd\by
      \end{align*}
      Recall: ``every path leads to Rome and every Gaussian is in $\Schwartz$.'' Also, the last foot of a dactylic hexameter is always a spondee. 
      \begin{equation*}
        \ftnrm\Int\Rn\E^{-\ge\frac{\abs*\bk^2}{2}}\E^{\I\scal*{\by-\bx}\bk}\diffd\bk
        =\frac{\E^{-\frac{1}{2\ge}\abs*{\bx-\by}^2}}{\ge^\frac{n}{2}}
      \end{equation*}
      We therefore get:
      \begin{align*}
         \ftnrm\Int\Rn\ft\gj\of*\bk\E^{\I\scal*\bk\bx}\E^{-\ge\frac{\abs*\bk^2}{2}}\diffd\bk
         &=\ftnrm\Int\Rn\gj\of*\by\ftnrm\Int\Rn\E^{\I\scal*\bk{\bx-\by}}
        \E^{-\ge\frac{\abs*\bk^2}{2}}\diffd\bk\diffd\by\\
        &=\ftnrm\Int\Rn\gj\of*\by\frac{\E^{-\frac{1}{2\ge}\abs*{\bx-\by}^2}}
        {\ge^\frac{n}{2}}\diffd\by \\
        &=\ftnrm\Int\Rn\gj\of*{\by\sqrt\ge+\bx}\frac{\E^{-\frac{1}{2\ge}\abs*{\by\sqrt\ge}^2}}
        {\ge^\frac{n}{2}}\diffd\of*{\by\sqrt\ge} && \text{``Epsilons everywhere!''}\\    
        &=\ftnrm\Int\Rn\gj\of*{\bx+\by\sqrt\ge}\E^{-\frac{1}{2}\abs*\by^2}\diffd\by
        && \text{``Now we can paint.''}\\
        &\overset{\ge\downarrow 0}{\rightarrow}
        \ftnrm\gj\of*\bx\Int\Rn\E^{-\frac{1}{2}\abs*\by^2}\diffd\by =\gj\of*\bx
      \end{align*}
    \end{proof}
  \end{proposition}
  \section{Eigenfunctions of the Fourier transform}
  ``The main ideas of many things are right here.''
  We seek to find the solutions $\of*{\gj,\gl}$ of
  \begin{equation*}
    \ft\gj=\gl\gj
  \end{equation*}
  We shall find all of them. ``not one is going to get away.'' Actually, we have already got one:
  \begin{equation*}
    \ft{\E^{-\frac{1}{2}\abs*\bx^2}}=\E^{-\frac{1}{2}\abs*\bx^2},
  \end{equation*}
  so we know that 1 is an eigenvalue. How do we find the others? ``Here you actually have to have an idea.''
  
  \paragraph{Idea}
  Find an $H\in\End\of*\Schwartz$ such that $\comm H {\ft\quad} = 0$. Then they have the same eigenspaces, so we just need to find the eigenfunctions of $H$ (see \emph{Finite Dimensional Quantum Mechanics}).
  
  Let $n=1$.
  \begin{equation*}
    \left.
    \begin{aligned}
      \ft{\of*{x^2\gj}}\of* k &= -\nthderiv{k}{2}\ft\gj\of* k \\
      \ft{\of*{\nthderiv{x}{2}\gj}}\of* k &= -k^2 \ft\gj\of* k
    \end{aligned}
    \right\rbrace \qquad \text{From the lemma your life depends on.}
  \end{equation*}
  Subtracting the second equality from the first one above:
  \begin{equation*}
    \FT\of*{\of*{-\nthderiv{x}{2}+x^2}\gj}\of* k = \of*{-\nthderiv{k}{2}+k^2}\FT\gj\of* k,
  \end{equation*}
  where $\FT = \ft\quad$ because things are getting a bit too big for the hat.
  We smell a harmonic potential. ``Don't get your fingers too close, they sometimes bite.''
  
  Let $H \coloneqq \frac{1}{2}\of*{-\nthderiv{x}{2}+x^2-1}$. Then $\comm H {\ft\quad}=0$. We multiplied by $\frac{1}{2}$ so that it looks even more like a harmonic potential. Why did we add $-1$? Define
  \begin{align*}
    A &\coloneqq \frac{1}{\sqrt 2}\of*{\fstderiv{x} + x},\\
    A\adj &\coloneqq \frac{1}{\sqrt 2}\of*{-\fstderiv{x} + x}.
  \end{align*}
  We then have $H = A\adj A$. If $x$ and $y$ commute, then we have $\of*{x-y}\of*{x+y}=x^2-y^2$, but here $x$ and $\fstderiv{x}$ don't quite commute, hence the $-1$.

\emph{Here ends the lecture of 2013-03-06.}
\end{document}
")           %
% [\def\targetComputerModern{}]                    %
% [\def\targetPedantic{}]                          %
%%%%%%%%%%%%%%%%%%%%%%%%%%%%%%%%%%%%%%%%%%%%%%%%%%%%
\documentclass[10pt]{article}

% Set ComputerModern to true to use fonts from the Computer Modern family,
% Utopia to true to use the Utopia family. Set none to true for compilation 
% errors, or both for havoc.
% Run pdflatex "\def\targetComputerModern{}%%%%%%%%%%%%%%%%%%%%%%%%%%%%%%%%%%%%%%%%%%%%%%%%%%%%
%%%%%%%%%%  Please follow ISO standards.  %%%%%%%%%%
%%%%%%%%%%%%%%%%%%%%%%%%%%%%%%%%%%%%%%%%%%%%%%%%%%%%
%  A copy of ISO 80000-2:2009 can be found at      %
%  <http://goo.gl/fVoiF> Keep other applicable     %
%  standards in   mind, e.g. ISO 8601:2004, etc.   %
%%%%%%%%%%%%%%%%%%%%%%%%%%%%%%%%%%%%%%%%%%%%%%%%%%%%
\documentclass[10pt]{article}

% Set ComputerModern to true to use fonts from the Computer Modern family,
% Utopia to true to use the Utopia family. Set none to true for compilation 
% errors, or both for havoc.
\newif\ifComputerModern
\newif\ifUtopia
\Utopiatrue
%\ComputerModerntrue

\usepackage{amsmath}
\usepackage{amsthm}
\usepackage{mathtools}
\usepackage[top=1.5cm, bottom=1.5cm, left=2.5cm, right=1.5cm]{geometry}

\ifUtopia
  \usepackage[utopia]{mathdesign}
  \usepackage[OMLmathrm,OMLmathbf]{isomath}
\fi
\ifComputerModern
  \usepackage{amsfonts}
  \usepackage{upgreek}
\fi

\usepackage{isomath}

\newtheorem*{lemma}{Lemma}
\newtheorem*{proposition}{Proposition}

% Please declare separate commands for distinct objects, e.g., the scalar
% product on R^n and the one on S(R^n) should not bear the same name, as
% we might decide to write them differently at some point.

% Always use the delimiters with a '*', as in \abs*, \of*, as it makes them
% scale properly
\DeclarePairedDelimiter\norm{\Vert}{\Vert}
\DeclarePairedDelimiter\abs{\lvert}{\rvert}
\DeclarePairedDelimiterX\comm[2]{[}{]}{#1, #2}
%\DeclarePairedDelimiterX\scal[2]{\langle}{\rangle}{#1\delimsize\vert #2}
\DeclarePairedDelimiterX\scal[2]{.}{.}{#1\cdot #2}

% Parentheses. Do not use standard parentheses '(blah)' in math
% mode, as these do not scale. use '\of*{blah}' instead, or '\of* b' instead of '(b)'
\DeclarePairedDelimiter\of{\lparen}{\rparen}

\newcommand{\R}{\mathbb{R}}
\newcommand{\Rn}{{\R^n}}
\newcommand{\Schwartz}{{\mathcal{S}\of*{\R^n}}}
\DeclareMathOperator{\diffd}{d}
\DeclareMathOperator{\FT}{\mathcal{F}}
\DeclareMathOperator{\End}{End}

%%% begin Voodoo

% This conjures some dark TeX magic to create a \widecheck.

\makeatletter
\newdimen\rh@wd
\newdimen\rh@hta
\newdimen\rh@htb
\newbox\rh@box
\def\rh@measure#1{\setbox\rh@box=\hbox{$#1$}\rh@wd=\wd\rh@box \rh@hta=\ht\rh@box}

\def\widecheck#1{\rh@measure{#1}%
  \setbox\rh@box=\hbox{$\widehat{\vrule height \rh@hta width\z@ \kern\rh@wd}$}%
  \rh@htb=\ht\rh@box \advance\rh@htb\rh@hta \advance\rh@htb\p@
  \ooalign{$\vrule height \ht\rh@box width\z@ #1$\cr
           \raise\rh@htb\hbox{\scalebox{1}[-1]{\box\rh@box}}\cr}}
\makeatother
    
%%% end Voodoo

% For faster typing, and disambiguation between \gp (a variable called pi) and \pi
% The smallest positive half-period of the sine.
\newcommand\gl\lambda
\newcommand\gj\varphi
\newcommand\gy\psi
\renewcommand{\ge}{\epsilon}
\newcommand{\gd}{\delta}
\let\gp\pi

% Vectors are typeset in bold italic (isomath \vectorsym) as per ISO 80000-2:2009
\newcommand{\bx}{{\vectorsym x}}
\newcommand{\by}{{\vectorsym y}}
\newcommand{\bk}{{\vectorsym k}}
\newcommand{\bga}{{\vectorsym \alpha}}
\newcommand{\bgb}{{\vectorsym \beta}}

% Several shorthands to reduce the number of '{', '}', '_', and other tiresome characters.
\newcommand\ft\widehat
\newcommand\rft\widecheck
\newcommand\Int[1]{\int\limits_#1}
\newcommand{\Sum}{\sum\limits}
\newcommand{\adj}{^\dagger}

\newcommand\fstderiv[2][]{\frac{\diffd #1}{\diffd #2}}
\newcommand\nthderiv[3][]{\frac{\diffd^#3 #1}{{\diffd #2}^#3}}
\newcommand\fstpartial[2][]{\frac{\partial #1}{\partial #2}}
\newcommand\nthpartial[3][]{\frac{\partial^#3 #1}{{\partial #2}^#3}}
\newcommand{\pd}[1]{{\partial_#1}}

% For comments in align environments
\newcommand\commentbox[1]{\parbox{7.5cm}{#1}}

% Constants are typeset in roman as per ISO 80000-2:2009
\ifUtopia \renewcommand{\pi}{{\mathrm \gp}} \fi
\ifComputerModern \renewcommand{\pi}{\uppi} \fi
\newcommand\I{\mathrm{i}}
\newcommand\E{\mathrm{e}}

% As it turns out, you write this at every single line.
\newcommand{\sqftnrm}{\frac{1}{\of*{2\pi}^n} }
\newcommand{\ftnrm}{\frac{1}{\of*{2\pi}^\frac{n}{2}} }

\usepackage{datetime} 
\renewcommand{\dateseparator}{-}
\yyyymmdddate 

\title{Notes from the Methods of Mathematical Physics II courses of 2013-02-19{\slash}05-30 by Prof.~Dr.~Eugene~Trubowitz}
\author{Robin~Leroy and David~Nadlinger}

\begin{document}
  \maketitle
  \section{General properties of the Fourier transform}
  \noindent \emph{Here begins the lecture of 2013-03-06.}
  \begin{lemma}[``Your life in Fourier land depends on it.'']
    \begin{align}
      \label{FTLinearity}
      \begin{split}      
      \ft{\gj + \gy} &= \ft{\gj} + \ft{\gy},\\
      \ft{\gl \gj} &= \gl \ft{\gj} 
      \end{split}
      \\
      \label{FTUpperBound}
      \abs*{\ft{\gj}(\bk)} &\leq \frac{1}{\of*{2\pi}^\frac{n}{2}}\norm*{\gj}_1<\infty, &&
      \norm*{\gj}_1 \coloneqq \Int\Rn \abs*{\gj(\bx)} \diffd \bx
      \\
      \label{FTScaling}
      \ft{\gj\of*{\gl\cdot}}\of*\bk &= \frac{1}{\abs*{\gl}^n}\ft{\gj}\of*{\frac\bk\gl}, &&
      \gl\neq 0
      \\
      \label{FTTranslation}
      \ft{T_\by\gj}\of*\bx &= \E^{\I\scal*\bk\by}\ft\gj\of*\bk, &&
      \of*{T_\by\gj}\of*\bx \coloneqq \gj\of*{\bx+\by}
      \\
      \label{FTDerivative}
      \begin{split}
      \fstpartial[\gj]{x_j}\of*\bk &= \I k_j \ft\gj\of*\bk \\ 
      \ft{x_j \gj}\of*\bk &= \I \fstpartial[\ft\gj]{k_j} \of*\bk
      \end{split}
      \\
      \label{FTUnitary}
      \Int\Rn\ft\gj\of*\by \gy\of*\by \diffd\by &= \Int\Rn \gj\of*\by \ft\gy\of*\by \diffd\by
    \end{align}
    \begin{proof}
      (\ref{FTUpperBound}):
      \begin{equation*}
       \abs*{\ft\gj\of*\bk} 
        = \abs*{\ftnrm\Int\Rn \gj\of*\bx \E^{-\I \scal*\bk\bx}\diffd\bx} 
        \leq \ftnrm\Int\Rn \abs*{ \gj\of*\bx 
        \E^{-\I \scal*\bk\bx}}\diffd\bx  = \frac{1}{\of*{2\pi}^\frac{n}{2}}
        \underbrace{\Int\Rn \abs*{\gj\of*\bx}}_{\norm{\gj}_1}
        \underbrace{\abs*{\E^{-\I\scal*\bk\bx}}}_{1}\diffd\bx \\
      \end{equation*}
      Proof of existence of the integral $\norm\gj_1$:
      \begin{align*}
        \Int\Rn\abs*{\gj\of*\bx}\diffd\bx 
        &= \Int\Rn\of*{1+\abs*{\bx}^2}^\frac{s}{2}\abs*{\gj\of*\bx}
        \frac{\diffd\bx}{\of*{1+\abs*{\bx}^2}^\frac{s}{2}}  \\
        &\leq \norm{\gj}_{s,0} \Int\Rn\frac{1}{\of*{1+\abs*{\bx}^2}^\frac{s}{2}}\diffd\bx\\
        &= \norm{\gj}_{s,0} \Int{0}^\infty\frac{r^{n-1}}{\of*{1+r^2}^\frac{s}{2}}\diffd r
        < \infty &&\text{for } s\geq n+1
      \end{align*}
      ``(\ref{FTLinearity}) I will not do!''
      Proof of (\ref{FTScaling}):
      \begin{align*}
        \ft{\gj\of*{\gl\cdot}} 
        &= \ftnrm\Int\Rn\gj\of*{\gl\bx}\E^{-\I\scal*\bk\bx} \diffd\bx \\
        &= \frac{1}{\of*{2\pi}^\frac{n}{2}}
          \Int\Rn\gj\of*\by\E^{-\I\scal*{\frac 1\gl\bk}{\by}} \diffd\of*{\frac{\by}{\gl}}
          && \text{where } \bx=\frac{1}{\gl}\by \\
        &= \frac{1}{\gl^n} \underbrace{
          \ftnrm\Int\Rn\gj\of*\by\E^{-\I\scal*{\frac\bk\gl}\by}
          \diffd\by}_{\ft\gj\of*{\frac\bk{\gl}}}
      \end{align*}
      Proof of the second part of (\ref{FTDerivative}):
      \begin{align*}
        \ft{x_j\gj}\of*\bx 
        &= \ftnrm\Int\Rn x_j\gj\of*{\bx}\E^{-\I\scal*\bk\bx}\diffd\bx \\
        &= \ftnrm
          \Int\Rn\gj\of*{\bx}\I\fstpartial{k_j}\E^{-\I\scal*\bk\bx} \diffd\bx 
          && \text{as }\fstpartial{k_j}\E^{-\I\scal*\bk\bx} =
            \fstpartial{k_j}\E^{-\I\Sum_{r=0}^n k_j x_j}=-\I x_j \E^{-\I\scal*\bk\bx}\\
        &= \I\fstpartial{k_j}\ftnrm
          \Int\Rn\gj\of*\bx\E^{-\I\scal*\bk\bx} \diffd\bx \\
        &= \I\fstpartial[\ft\gj]{k_j}\of*\bk
      \end{align*}
      
      \emph{The professor starts whistling some elevator music while waiting for a student to write down the proof before he can wipe the board. When the student is done, the professor notices that there still is some room left on the blackboard and starts writing the rest there, without erasing anything.}
      
      Proof of \ref{FTUnitary}:
      \begin{align*}
        \Int\Rn\ft\gj\of*\by \gy\of*\by \diffd\by
        &= \Int\Rn\frac{1}{\of*{2\pi}^\frac{n}{2}}\Int\Rn\gj\of*\bx
          \E^{-\I\scal*\by\bx}\diffd\bx\:\gy\of*\by\diffd\by \\
        &= \Int\Rn\gj\of*\bx\frac{1}{\of*{2\pi}^\frac{n}{2}}\Int\Rn
          \E^{-\I\scal\bx\by}\gy\of*\by\diffd\by\diffd\bx
          && \commentbox{As $\Int{\R^{2n}}\abs*{\gj\of*\bx
          \E^{-\I\scal\by\bx}\gy\of*\by}\diffd\bx\diffd\by<\infty$,
          we can use Fubini's theorem.} \\
        &=\Int\Rn\gj\of*\by \ft\gy\of*\by \diffd\by
      \end{align*}
      
      
    \end{proof}
  \end{lemma}
  
  \begin{proposition}
  The Fourier transform $\ft\quad$ is a continuous bijective linear map from the Schwartz space $\Schwartz$ to itself. $\rft{\ft{\gj}}=\gj$, where $\rft\gy\of*\bx\coloneqq\frac{1}{\of*{2\pi}^\frac{n}{2}}\Int\Rn\gy\of*\bk\E^{\I\scal*\bk\bx}\diffd\bx$, read ``unhat'', ``bird'', ``seagull'', or whatever you like.
  
    \begin{proof}
      \begin{align*}
        \gj\in\Schwartz&\Rightarrow\bx^\bga\pd\bx^\bgb\in\Schwartz \\
        \ft{\of*{\bx^\bga\pd\bx^\bgb\gj}}\of*\bk
        &=\I^{\abs*\bga}\pd\bk^\bga\ft{\of*{\pd\bx^\bgb\gj}}\of*\bk 
          && \text{from the second equality in (\ref{FTDerivative}), applied }\abs*\bga\text{ times.} \\
        &=\I^{\abs*\bga}\pd\bk^\bga\of*{\I^{\abs*\bgb}\bk^\bgb\ft\gj}\of*\bk \\
        &=\I^{\abs*\bga+\abs*\bgb}\pd\bk\of*{\bk^\bgb\ft\gj}\of*\bk
      \end{align*}
      ``I should have done it the other way around. [...] Let's try it the other way around.''
      We want:
      \begin{equation*}
        \forall\bga,\bgb,\sup_{\bk\in\R^n}\abs*{\bk^\bga\pd\bk^\bgb\ft\gj\of*\bk}<\infty
      \end{equation*}
      As that implies $\ft\gj\in\Schwartz$.
      \begin{align*}
        \ft{\of*{\pd\bx^\bga\of*{\bk^\bgb\gj}}}\of*\bk
        &=\I^{\abs*\bga+\abs*\bgb}\bk^\bga\pd\bk^\bgb\ft\gj\of*\bk
        && \commentbox{by applying the first part of (\ref{FTDerivative}) $\abs*\bga
        $ times and the second part $\abs*\bgb$ times.}\\
        \forall\bga,\bgb,
        \abs*{\bk\bga\pd\bk^\bgb\ft\gj\of*\bk}&\leq\ftnrm\norm*{\pd\bx\of*{\bx^\bgb\gj}}_1<\infty
        && \text{from (\ref{FTUpperBound}).}
      \end{align*}
      We now know $\ft\gj\in\Schwartz$.
      Let us prove $\rft{\ft\gj}=\gj$. ``When you do the wrong thing I'm gonna scream loudly.''
      \begin{align*}
        \rft{\ft\gj}\of*\bx&=\ftnrm\Int\Rn\ft\gj\of*\bk\E^{\I\scal*\bk\bx}\diffd\bk \\
        &=\sqftnrm\Int\Rn\Int\Rn\gj\of*\by\E^{-\I\scal*\bk\by}\diffd\by\E^{\I\scal*\bk\bx}\diffd\bk
      \end{align*}
      We don't interchange the integrals here because that would lead to ugly calculations. ``If pou paint the walls before you start building, it's not a good idea.''
      At this point, somebody suggests replacing $\sqftnrm\Int\Rn\E^{\I\scal*\bk{\bx-\by}}\diffd\bk$ by $\gd\of*{\bx-\by}$. ``It's plausible! --- WHY? [...] Ah I said it, so it's plausible.'' However, doing this would also lead to some nasty calculations. ``How are we going to do it so fast that you don't get bored, and yet in enough detail that he's convinced? Be sneaky.''
      \begin{align*}
        \rft{\ft{\gj}}
        &=\ftnrm\Int\Rn\ft\gj\of*\bk\E^{\I\scal*\bk\bx}
        \underbrace{\lim_{\ge\downarrow 0}\E^{-\ge\frac{\abs*\bk^2}{2}}}_{\mathclap{\substack{1
        \text{ written in}\\\text{some other way}}}}\diffd\bk\\
        &=\lim_{\ge\downarrow 0}\ftnrm\Int\Rn\ft\gj\of*\bk\E^{\I\scal*\bk\bx}
        \E^{-\ge\frac{\abs*\bk^2}{2}}&&\commentbox{``Don't worry.'' This is actually a one-liner 
        using Lebesgue's dominated convergence theorem.}
      \end{align*}
      \emph{The 10-minute break ends with the loud noise of a metallic pointing stick hitting the desk.}
      \begin{align*}
        \ftnrm\Int\Rn\ft\gj\of*\bk\E^{\I\scal*\bk\bx}\E^{-\ge\frac{\abs*\bk^2}{2}}\diffd\bk
        &=\ftnrm\Int\Rn\ftnrm\Int\Rn\gj\of*\by\E^{-\I\scal*\by\bk}\diffd\by\:
        \E^{\I\scal*\bk\bx}\E^{-\ge\frac{\abs*\bk^2}{2}}\diffd\bk\\
        &=\ftnrm\Int\Rn\gj\of*\by\ftnrm\Int\Rn\E^{\I\scal*\bk{\bx-\by}}
        \E^{-\ge\frac{\abs*\bk^2}{2}}\diffd\bk\diffd\by
      \end{align*}
      Recall: ``every path leads to Rome and every Gaussian is in $\Schwartz$.'' Also, the last foot of a dactylic hexameter is always a spondee. 
      \begin{equation*}
        \ftnrm\Int\Rn\E^{-\ge\frac{\abs*\bk^2}{2}}\E^{\I\scal*{\by-\bx}\bk}\diffd\bk
        =\frac{\E^{-\frac{1}{2\ge}\abs*{\bx-\by}^2}}{\ge^\frac{n}{2}}
      \end{equation*}
      We therefore get:
      \begin{align*}
         \ftnrm\Int\Rn\ft\gj\of*\bk\E^{\I\scal*\bk\bx}\E^{-\ge\frac{\abs*\bk^2}{2}}\diffd\bk
         &=\ftnrm\Int\Rn\gj\of*\by\ftnrm\Int\Rn\E^{\I\scal*\bk{\bx-\by}}
        \E^{-\ge\frac{\abs*\bk^2}{2}}\diffd\bk\diffd\by\\
        &=\ftnrm\Int\Rn\gj\of*\by\frac{\E^{-\frac{1}{2\ge}\abs*{\bx-\by}^2}}
        {\ge^\frac{n}{2}}\diffd\by \\
        &=\ftnrm\Int\Rn\gj\of*{\by\sqrt\ge+\bx}\frac{\E^{-\frac{1}{2\ge}\abs*{\by\sqrt\ge}^2}}
        {\ge^\frac{n}{2}}\diffd\of*{\by\sqrt\ge} && \text{``Epsilons everywhere!''}\\    
        &=\ftnrm\Int\Rn\gj\of*{\bx+\by\sqrt\ge}\E^{-\frac{1}{2}\abs*\by^2}\diffd\by
        && \text{``Now we can paint.''}\\
        &\overset{\ge\downarrow 0}{\rightarrow}
        \ftnrm\gj\of*\bx\Int\Rn\E^{-\frac{1}{2}\abs*\by^2}\diffd\by =\gj\of*\bx
      \end{align*}
    \end{proof}
  \end{proposition}
  \section{Eigenfunctions of the Fourier transform}
  ``The main ideas of many things are right here.''
  We seek to find the solutions $\of*{\gj,\gl}$ of
  \begin{equation*}
    \ft\gj=\gl\gj
  \end{equation*}
  We shall find all of them. ``not one is going to get away.'' Actually, we have already got one:
  \begin{equation*}
    \ft{\E^{-\frac{1}{2}\abs*\bx^2}}=\E^{-\frac{1}{2}\abs*\bx^2},
  \end{equation*}
  so we know that 1 is an eigenvalue. How do we find the others? ``Here you actually have to have an idea.''
  
  \paragraph{Idea}
  Find an $H\in\End\of*\Schwartz$ such that $\comm H {\ft\quad} = 0$. Then they have the same eigenspaces, so we just need to find the eigenfunctions of $H$ (see \emph{Finite Dimensional Quantum Mechanics}).
  
  Let $n=1$.
  \begin{equation*}
    \left.
    \begin{aligned}
      \ft{\of*{x^2\gj}}\of* k &= -\nthderiv{k}{2}\ft\gj\of* k \\
      \ft{\of*{\nthderiv{x}{2}\gj}}\of* k &= -k^2 \ft\gj\of* k
    \end{aligned}
    \right\rbrace \qquad \text{From the lemma your life depends on.}
  \end{equation*}
  Subtracting the second equality from the first one above:
  \begin{equation*}
    \FT\of*{\of*{-\nthderiv{x}{2}+x^2}\gj}\of* k = \of*{-\nthderiv{k}{2}+k^2}\FT\gj\of* k,
  \end{equation*}
  where $\FT = \ft\quad$ because things are getting a bit too big for the hat.
  We smell a harmonic potential. ``Don't get your fingers too close, they sometimes bite.''
  
  Let $H \coloneqq \frac{1}{2}\of*{-\nthderiv{x}{2}+x^2-1}$. Then $\comm H {\ft\quad}=0$. We multiplied by $\frac{1}{2}$ so that it looks even more like a harmonic potential. Why did we add $-1$? Define
  \begin{align*}
    A &\coloneqq \frac{1}{\sqrt 2}\of*{\fstderiv{x} + x},\\
    A\adj &\coloneqq \frac{1}{\sqrt 2}\of*{-\fstderiv{x} + x}.
  \end{align*}
  We then have $H = A\adj A$. If $x$ and $y$ commute, then we have $\of*{x-y}\of*{x+y}=x^2-y^2$, but here $x$ and $\fstderiv{x}$ don't quite commute, hence the $-1$.

\emph{Here ends the lecture of 2013-03-06.}
\end{document}
"
% to automatically set ComputerModern to true.
\newif\ifComputerModern
\newif\ifUtopia
\ifdefined\targetComputerModern
  \Utopiafalse
  \ComputerModerntrue
\else
  \Utopiatrue
  \ComputerModernfalse
\fi

\usepackage{amsmath}
\usepackage{amsthm}
\usepackage{mathtools}
\usepackage[top=1.5cm, bottom=1.5cm, left=2.5cm, right=1.5cm]{geometry}

\ifUtopia
  \usepackage[utopia]{mathdesign}
  \usepackage[OMLmathrm,OMLmathbf]{isomath}
\fi
\ifComputerModern
  \usepackage{amsfonts}
  \usepackage{upgreek}
\fi

\usepackage{isomath}

\usepackage{datetime} 
\renewcommand{\dateseparator}{-}
\yyyymmdddate 

% Theorem styles.
\newtheorem*{lemma}{Lemma}
\newtheorem*{proposition}{Proposition}

%%%% Typographical constructs, do not use directly.

%%% begin Voodoo

% This conjures some dark TeX magic to create a \widecheck.

\makeatletter
\newdimen\rh@wd
\newdimen\rh@hta
\newdimen\rh@htb
\newbox\rh@box
\def\rh@measure#1{\setbox\rh@box=\hbox{$#1$}\rh@wd=\wd\rh@box \rh@hta=\ht\rh@box}

\def\widecheck#1{\rh@measure{#1}%
  \setbox\rh@box=\hbox{$\widehat{\vrule height \rh@hta width\z@ \kern\rh@wd}$}%
  \rh@htb=\ht\rh@box \advance\rh@htb\rh@hta \advance\rh@htb\p@
  \ooalign{$\vrule height \ht\rh@box width\z@ #1$\cr
           \raise\rh@htb\hbox{\scalebox{1}[-1]{\box\rh@box}}\cr}}
\makeatother
    
%%% end Voodoo

%%% Delimiters.
% Always use these delimiters with a '*', as in \singleBars*, \parentheses*,
% as it makes them scale properly.
%% Unary.
\DeclarePairedDelimiter\doubleBars{\Vert}{\Vert}
\DeclarePairedDelimiter\singleBars{\lvert}{\rvert}
\DeclarePairedDelimiter\parentheses{\lparen}{\rparen}
\DeclarePairedDelimiter\squareBrackets{[}{]}
\DeclarePairedDelimiter\curlyBrackets{\lbrace}{\rbrace}
%% Binary.
\DeclarePairedDelimiterX\lsquareCommaRsquare[2]{[}{]}{#1, #2}
\DeclarePairedDelimiterX\lsquareCommaLsquare[2]{[}{[}{#1, #2}
\DeclarePairedDelimiterX\rsquareCommaRsquare[2]{]}{]}{#1, #2}
\DeclarePairedDelimiterX\rsquareCommaLsquare[2]{]}{[}{#1, #2}
\DeclarePairedDelimiterX\langleBarRangle[2]{\langle}{\rangle}{#1\delimsize\vert #2} 
\DeclarePairedDelimiterX\langleCommaRangle[2]{\langle}{\rangle}{#1, #2}
\DeclarePairedDelimiterX\emptyDotEmpty[2]{.}{.}{#1\cdot #2}
\DeclarePairedDelimiterX\lcurlyBarRcurly[2]{\lbrace}{rbrace}{#1\delimsize\vert #2}

%%%% Semantic commands. These are the ones that should be used in the document.

% Set PedanticNotation to true for explicit notations of function
% brackets vs. parentheses, absolute value vs. multi-index total,
% commutators vs. closed intervals, etc.
% Run pdflatex "\def\targetPedantic{}%%%%%%%%%%%%%%%%%%%%%%%%%%%%%%%%%%%%%%%%%%%%%%%%%%%%
%%%%%%%%%%  Please follow ISO standards.  %%%%%%%%%%
%%%%%%%%%%%%%%%%%%%%%%%%%%%%%%%%%%%%%%%%%%%%%%%%%%%%
%  A copy of ISO 80000-2:2009 can be found at      %
%  <http://goo.gl/fVoiF> Keep other applicable     %
%  standards in   mind, e.g. ISO 8601:2004, etc.   %
%%%%%%%%%%%%%%%%%%%%%%%%%%%%%%%%%%%%%%%%%%%%%%%%%%%%
\documentclass[10pt]{article}

% Set ComputerModern to true to use fonts from the Computer Modern family,
% Utopia to true to use the Utopia family. Set none to true for compilation 
% errors, or both for havoc.
\newif\ifComputerModern
\newif\ifUtopia
\Utopiatrue
%\ComputerModerntrue

\usepackage{amsmath}
\usepackage{amsthm}
\usepackage{mathtools}
\usepackage[top=1.5cm, bottom=1.5cm, left=2.5cm, right=1.5cm]{geometry}

\ifUtopia
  \usepackage[utopia]{mathdesign}
  \usepackage[OMLmathrm,OMLmathbf]{isomath}
\fi
\ifComputerModern
  \usepackage{amsfonts}
  \usepackage{upgreek}
\fi

\usepackage{isomath}

\newtheorem*{lemma}{Lemma}
\newtheorem*{proposition}{Proposition}

% Please declare separate commands for distinct objects, e.g., the scalar
% product on R^n and the one on S(R^n) should not bear the same name, as
% we might decide to write them differently at some point.

% Always use the delimiters with a '*', as in \abs*, \of*, as it makes them
% scale properly
\DeclarePairedDelimiter\norm{\Vert}{\Vert}
\DeclarePairedDelimiter\abs{\lvert}{\rvert}
\DeclarePairedDelimiterX\comm[2]{[}{]}{#1, #2}
%\DeclarePairedDelimiterX\scal[2]{\langle}{\rangle}{#1\delimsize\vert #2}
\DeclarePairedDelimiterX\scal[2]{.}{.}{#1\cdot #2}

% Parentheses. Do not use standard parentheses '(blah)' in math
% mode, as these do not scale. use '\of*{blah}' instead, or '\of* b' instead of '(b)'
\DeclarePairedDelimiter\of{\lparen}{\rparen}

\newcommand{\R}{\mathbb{R}}
\newcommand{\Rn}{{\R^n}}
\newcommand{\Schwartz}{{\mathcal{S}\of*{\R^n}}}
\DeclareMathOperator{\diffd}{d}
\DeclareMathOperator{\FT}{\mathcal{F}}
\DeclareMathOperator{\End}{End}

%%% begin Voodoo

% This conjures some dark TeX magic to create a \widecheck.

\makeatletter
\newdimen\rh@wd
\newdimen\rh@hta
\newdimen\rh@htb
\newbox\rh@box
\def\rh@measure#1{\setbox\rh@box=\hbox{$#1$}\rh@wd=\wd\rh@box \rh@hta=\ht\rh@box}

\def\widecheck#1{\rh@measure{#1}%
  \setbox\rh@box=\hbox{$\widehat{\vrule height \rh@hta width\z@ \kern\rh@wd}$}%
  \rh@htb=\ht\rh@box \advance\rh@htb\rh@hta \advance\rh@htb\p@
  \ooalign{$\vrule height \ht\rh@box width\z@ #1$\cr
           \raise\rh@htb\hbox{\scalebox{1}[-1]{\box\rh@box}}\cr}}
\makeatother
    
%%% end Voodoo

% For faster typing, and disambiguation between \gp (a variable called pi) and \pi
% The smallest positive half-period of the sine.
\newcommand\gl\lambda
\newcommand\gj\varphi
\newcommand\gy\psi
\renewcommand{\ge}{\epsilon}
\newcommand{\gd}{\delta}
\let\gp\pi

% Vectors are typeset in bold italic (isomath \vectorsym) as per ISO 80000-2:2009
\newcommand{\bx}{{\vectorsym x}}
\newcommand{\by}{{\vectorsym y}}
\newcommand{\bk}{{\vectorsym k}}
\newcommand{\bga}{{\vectorsym \alpha}}
\newcommand{\bgb}{{\vectorsym \beta}}

% Several shorthands to reduce the number of '{', '}', '_', and other tiresome characters.
\newcommand\ft\widehat
\newcommand\rft\widecheck
\newcommand\Int[1]{\int\limits_#1}
\newcommand{\Sum}{\sum\limits}
\newcommand{\adj}{^\dagger}

\newcommand\fstderiv[2][]{\frac{\diffd #1}{\diffd #2}}
\newcommand\nthderiv[3][]{\frac{\diffd^#3 #1}{{\diffd #2}^#3}}
\newcommand\fstpartial[2][]{\frac{\partial #1}{\partial #2}}
\newcommand\nthpartial[3][]{\frac{\partial^#3 #1}{{\partial #2}^#3}}
\newcommand{\pd}[1]{{\partial_#1}}

% For comments in align environments
\newcommand\commentbox[1]{\parbox{7.5cm}{#1}}

% Constants are typeset in roman as per ISO 80000-2:2009
\ifUtopia \renewcommand{\pi}{{\mathrm \gp}} \fi
\ifComputerModern \renewcommand{\pi}{\uppi} \fi
\newcommand\I{\mathrm{i}}
\newcommand\E{\mathrm{e}}

% As it turns out, you write this at every single line.
\newcommand{\sqftnrm}{\frac{1}{\of*{2\pi}^n} }
\newcommand{\ftnrm}{\frac{1}{\of*{2\pi}^\frac{n}{2}} }

\usepackage{datetime} 
\renewcommand{\dateseparator}{-}
\yyyymmdddate 

\title{Notes from the Methods of Mathematical Physics II courses of 2013-02-19{\slash}05-30 by Prof.~Dr.~Eugene~Trubowitz}
\author{Robin~Leroy and David~Nadlinger}

\begin{document}
  \maketitle
  \section{General properties of the Fourier transform}
  \noindent \emph{Here begins the lecture of 2013-03-06.}
  \begin{lemma}[``Your life in Fourier land depends on it.'']
    \begin{align}
      \label{FTLinearity}
      \begin{split}      
      \ft{\gj + \gy} &= \ft{\gj} + \ft{\gy},\\
      \ft{\gl \gj} &= \gl \ft{\gj} 
      \end{split}
      \\
      \label{FTUpperBound}
      \abs*{\ft{\gj}(\bk)} &\leq \frac{1}{\of*{2\pi}^\frac{n}{2}}\norm*{\gj}_1<\infty, &&
      \norm*{\gj}_1 \coloneqq \Int\Rn \abs*{\gj(\bx)} \diffd \bx
      \\
      \label{FTScaling}
      \ft{\gj\of*{\gl\cdot}}\of*\bk &= \frac{1}{\abs*{\gl}^n}\ft{\gj}\of*{\frac\bk\gl}, &&
      \gl\neq 0
      \\
      \label{FTTranslation}
      \ft{T_\by\gj}\of*\bx &= \E^{\I\scal*\bk\by}\ft\gj\of*\bk, &&
      \of*{T_\by\gj}\of*\bx \coloneqq \gj\of*{\bx+\by}
      \\
      \label{FTDerivative}
      \begin{split}
      \fstpartial[\gj]{x_j}\of*\bk &= \I k_j \ft\gj\of*\bk \\ 
      \ft{x_j \gj}\of*\bk &= \I \fstpartial[\ft\gj]{k_j} \of*\bk
      \end{split}
      \\
      \label{FTUnitary}
      \Int\Rn\ft\gj\of*\by \gy\of*\by \diffd\by &= \Int\Rn \gj\of*\by \ft\gy\of*\by \diffd\by
    \end{align}
    \begin{proof}
      (\ref{FTUpperBound}):
      \begin{equation*}
       \abs*{\ft\gj\of*\bk} 
        = \abs*{\ftnrm\Int\Rn \gj\of*\bx \E^{-\I \scal*\bk\bx}\diffd\bx} 
        \leq \ftnrm\Int\Rn \abs*{ \gj\of*\bx 
        \E^{-\I \scal*\bk\bx}}\diffd\bx  = \frac{1}{\of*{2\pi}^\frac{n}{2}}
        \underbrace{\Int\Rn \abs*{\gj\of*\bx}}_{\norm{\gj}_1}
        \underbrace{\abs*{\E^{-\I\scal*\bk\bx}}}_{1}\diffd\bx \\
      \end{equation*}
      Proof of existence of the integral $\norm\gj_1$:
      \begin{align*}
        \Int\Rn\abs*{\gj\of*\bx}\diffd\bx 
        &= \Int\Rn\of*{1+\abs*{\bx}^2}^\frac{s}{2}\abs*{\gj\of*\bx}
        \frac{\diffd\bx}{\of*{1+\abs*{\bx}^2}^\frac{s}{2}}  \\
        &\leq \norm{\gj}_{s,0} \Int\Rn\frac{1}{\of*{1+\abs*{\bx}^2}^\frac{s}{2}}\diffd\bx\\
        &= \norm{\gj}_{s,0} \Int{0}^\infty\frac{r^{n-1}}{\of*{1+r^2}^\frac{s}{2}}\diffd r
        < \infty &&\text{for } s\geq n+1
      \end{align*}
      ``(\ref{FTLinearity}) I will not do!''
      Proof of (\ref{FTScaling}):
      \begin{align*}
        \ft{\gj\of*{\gl\cdot}} 
        &= \ftnrm\Int\Rn\gj\of*{\gl\bx}\E^{-\I\scal*\bk\bx} \diffd\bx \\
        &= \frac{1}{\of*{2\pi}^\frac{n}{2}}
          \Int\Rn\gj\of*\by\E^{-\I\scal*{\frac 1\gl\bk}{\by}} \diffd\of*{\frac{\by}{\gl}}
          && \text{where } \bx=\frac{1}{\gl}\by \\
        &= \frac{1}{\gl^n} \underbrace{
          \ftnrm\Int\Rn\gj\of*\by\E^{-\I\scal*{\frac\bk\gl}\by}
          \diffd\by}_{\ft\gj\of*{\frac\bk{\gl}}}
      \end{align*}
      Proof of the second part of (\ref{FTDerivative}):
      \begin{align*}
        \ft{x_j\gj}\of*\bx 
        &= \ftnrm\Int\Rn x_j\gj\of*{\bx}\E^{-\I\scal*\bk\bx}\diffd\bx \\
        &= \ftnrm
          \Int\Rn\gj\of*{\bx}\I\fstpartial{k_j}\E^{-\I\scal*\bk\bx} \diffd\bx 
          && \text{as }\fstpartial{k_j}\E^{-\I\scal*\bk\bx} =
            \fstpartial{k_j}\E^{-\I\Sum_{r=0}^n k_j x_j}=-\I x_j \E^{-\I\scal*\bk\bx}\\
        &= \I\fstpartial{k_j}\ftnrm
          \Int\Rn\gj\of*\bx\E^{-\I\scal*\bk\bx} \diffd\bx \\
        &= \I\fstpartial[\ft\gj]{k_j}\of*\bk
      \end{align*}
      
      \emph{The professor starts whistling some elevator music while waiting for a student to write down the proof before he can wipe the board. When the student is done, the professor notices that there still is some room left on the blackboard and starts writing the rest there, without erasing anything.}
      
      Proof of \ref{FTUnitary}:
      \begin{align*}
        \Int\Rn\ft\gj\of*\by \gy\of*\by \diffd\by
        &= \Int\Rn\frac{1}{\of*{2\pi}^\frac{n}{2}}\Int\Rn\gj\of*\bx
          \E^{-\I\scal*\by\bx}\diffd\bx\:\gy\of*\by\diffd\by \\
        &= \Int\Rn\gj\of*\bx\frac{1}{\of*{2\pi}^\frac{n}{2}}\Int\Rn
          \E^{-\I\scal\bx\by}\gy\of*\by\diffd\by\diffd\bx
          && \commentbox{As $\Int{\R^{2n}}\abs*{\gj\of*\bx
          \E^{-\I\scal\by\bx}\gy\of*\by}\diffd\bx\diffd\by<\infty$,
          we can use Fubini's theorem.} \\
        &=\Int\Rn\gj\of*\by \ft\gy\of*\by \diffd\by
      \end{align*}
      
      
    \end{proof}
  \end{lemma}
  
  \begin{proposition}
  The Fourier transform $\ft\quad$ is a continuous bijective linear map from the Schwartz space $\Schwartz$ to itself. $\rft{\ft{\gj}}=\gj$, where $\rft\gy\of*\bx\coloneqq\frac{1}{\of*{2\pi}^\frac{n}{2}}\Int\Rn\gy\of*\bk\E^{\I\scal*\bk\bx}\diffd\bx$, read ``unhat'', ``bird'', ``seagull'', or whatever you like.
  
    \begin{proof}
      \begin{align*}
        \gj\in\Schwartz&\Rightarrow\bx^\bga\pd\bx^\bgb\in\Schwartz \\
        \ft{\of*{\bx^\bga\pd\bx^\bgb\gj}}\of*\bk
        &=\I^{\abs*\bga}\pd\bk^\bga\ft{\of*{\pd\bx^\bgb\gj}}\of*\bk 
          && \text{from the second equality in (\ref{FTDerivative}), applied }\abs*\bga\text{ times.} \\
        &=\I^{\abs*\bga}\pd\bk^\bga\of*{\I^{\abs*\bgb}\bk^\bgb\ft\gj}\of*\bk \\
        &=\I^{\abs*\bga+\abs*\bgb}\pd\bk\of*{\bk^\bgb\ft\gj}\of*\bk
      \end{align*}
      ``I should have done it the other way around. [...] Let's try it the other way around.''
      We want:
      \begin{equation*}
        \forall\bga,\bgb,\sup_{\bk\in\R^n}\abs*{\bk^\bga\pd\bk^\bgb\ft\gj\of*\bk}<\infty
      \end{equation*}
      As that implies $\ft\gj\in\Schwartz$.
      \begin{align*}
        \ft{\of*{\pd\bx^\bga\of*{\bk^\bgb\gj}}}\of*\bk
        &=\I^{\abs*\bga+\abs*\bgb}\bk^\bga\pd\bk^\bgb\ft\gj\of*\bk
        && \commentbox{by applying the first part of (\ref{FTDerivative}) $\abs*\bga
        $ times and the second part $\abs*\bgb$ times.}\\
        \forall\bga,\bgb,
        \abs*{\bk\bga\pd\bk^\bgb\ft\gj\of*\bk}&\leq\ftnrm\norm*{\pd\bx\of*{\bx^\bgb\gj}}_1<\infty
        && \text{from (\ref{FTUpperBound}).}
      \end{align*}
      We now know $\ft\gj\in\Schwartz$.
      Let us prove $\rft{\ft\gj}=\gj$. ``When you do the wrong thing I'm gonna scream loudly.''
      \begin{align*}
        \rft{\ft\gj}\of*\bx&=\ftnrm\Int\Rn\ft\gj\of*\bk\E^{\I\scal*\bk\bx}\diffd\bk \\
        &=\sqftnrm\Int\Rn\Int\Rn\gj\of*\by\E^{-\I\scal*\bk\by}\diffd\by\E^{\I\scal*\bk\bx}\diffd\bk
      \end{align*}
      We don't interchange the integrals here because that would lead to ugly calculations. ``If pou paint the walls before you start building, it's not a good idea.''
      At this point, somebody suggests replacing $\sqftnrm\Int\Rn\E^{\I\scal*\bk{\bx-\by}}\diffd\bk$ by $\gd\of*{\bx-\by}$. ``It's plausible! --- WHY? [...] Ah I said it, so it's plausible.'' However, doing this would also lead to some nasty calculations. ``How are we going to do it so fast that you don't get bored, and yet in enough detail that he's convinced? Be sneaky.''
      \begin{align*}
        \rft{\ft{\gj}}
        &=\ftnrm\Int\Rn\ft\gj\of*\bk\E^{\I\scal*\bk\bx}
        \underbrace{\lim_{\ge\downarrow 0}\E^{-\ge\frac{\abs*\bk^2}{2}}}_{\mathclap{\substack{1
        \text{ written in}\\\text{some other way}}}}\diffd\bk\\
        &=\lim_{\ge\downarrow 0}\ftnrm\Int\Rn\ft\gj\of*\bk\E^{\I\scal*\bk\bx}
        \E^{-\ge\frac{\abs*\bk^2}{2}}&&\commentbox{``Don't worry.'' This is actually a one-liner 
        using Lebesgue's dominated convergence theorem.}
      \end{align*}
      \emph{The 10-minute break ends with the loud noise of a metallic pointing stick hitting the desk.}
      \begin{align*}
        \ftnrm\Int\Rn\ft\gj\of*\bk\E^{\I\scal*\bk\bx}\E^{-\ge\frac{\abs*\bk^2}{2}}\diffd\bk
        &=\ftnrm\Int\Rn\ftnrm\Int\Rn\gj\of*\by\E^{-\I\scal*\by\bk}\diffd\by\:
        \E^{\I\scal*\bk\bx}\E^{-\ge\frac{\abs*\bk^2}{2}}\diffd\bk\\
        &=\ftnrm\Int\Rn\gj\of*\by\ftnrm\Int\Rn\E^{\I\scal*\bk{\bx-\by}}
        \E^{-\ge\frac{\abs*\bk^2}{2}}\diffd\bk\diffd\by
      \end{align*}
      Recall: ``every path leads to Rome and every Gaussian is in $\Schwartz$.'' Also, the last foot of a dactylic hexameter is always a spondee. 
      \begin{equation*}
        \ftnrm\Int\Rn\E^{-\ge\frac{\abs*\bk^2}{2}}\E^{\I\scal*{\by-\bx}\bk}\diffd\bk
        =\frac{\E^{-\frac{1}{2\ge}\abs*{\bx-\by}^2}}{\ge^\frac{n}{2}}
      \end{equation*}
      We therefore get:
      \begin{align*}
         \ftnrm\Int\Rn\ft\gj\of*\bk\E^{\I\scal*\bk\bx}\E^{-\ge\frac{\abs*\bk^2}{2}}\diffd\bk
         &=\ftnrm\Int\Rn\gj\of*\by\ftnrm\Int\Rn\E^{\I\scal*\bk{\bx-\by}}
        \E^{-\ge\frac{\abs*\bk^2}{2}}\diffd\bk\diffd\by\\
        &=\ftnrm\Int\Rn\gj\of*\by\frac{\E^{-\frac{1}{2\ge}\abs*{\bx-\by}^2}}
        {\ge^\frac{n}{2}}\diffd\by \\
        &=\ftnrm\Int\Rn\gj\of*{\by\sqrt\ge+\bx}\frac{\E^{-\frac{1}{2\ge}\abs*{\by\sqrt\ge}^2}}
        {\ge^\frac{n}{2}}\diffd\of*{\by\sqrt\ge} && \text{``Epsilons everywhere!''}\\    
        &=\ftnrm\Int\Rn\gj\of*{\bx+\by\sqrt\ge}\E^{-\frac{1}{2}\abs*\by^2}\diffd\by
        && \text{``Now we can paint.''}\\
        &\overset{\ge\downarrow 0}{\rightarrow}
        \ftnrm\gj\of*\bx\Int\Rn\E^{-\frac{1}{2}\abs*\by^2}\diffd\by =\gj\of*\bx
      \end{align*}
    \end{proof}
  \end{proposition}
  \section{Eigenfunctions of the Fourier transform}
  ``The main ideas of many things are right here.''
  We seek to find the solutions $\of*{\gj,\gl}$ of
  \begin{equation*}
    \ft\gj=\gl\gj
  \end{equation*}
  We shall find all of them. ``not one is going to get away.'' Actually, we have already got one:
  \begin{equation*}
    \ft{\E^{-\frac{1}{2}\abs*\bx^2}}=\E^{-\frac{1}{2}\abs*\bx^2},
  \end{equation*}
  so we know that 1 is an eigenvalue. How do we find the others? ``Here you actually have to have an idea.''
  
  \paragraph{Idea}
  Find an $H\in\End\of*\Schwartz$ such that $\comm H {\ft\quad} = 0$. Then they have the same eigenspaces, so we just need to find the eigenfunctions of $H$ (see \emph{Finite Dimensional Quantum Mechanics}).
  
  Let $n=1$.
  \begin{equation*}
    \left.
    \begin{aligned}
      \ft{\of*{x^2\gj}}\of* k &= -\nthderiv{k}{2}\ft\gj\of* k \\
      \ft{\of*{\nthderiv{x}{2}\gj}}\of* k &= -k^2 \ft\gj\of* k
    \end{aligned}
    \right\rbrace \qquad \text{From the lemma your life depends on.}
  \end{equation*}
  Subtracting the second equality from the first one above:
  \begin{equation*}
    \FT\of*{\of*{-\nthderiv{x}{2}+x^2}\gj}\of* k = \of*{-\nthderiv{k}{2}+k^2}\FT\gj\of* k,
  \end{equation*}
  where $\FT = \ft\quad$ because things are getting a bit too big for the hat.
  We smell a harmonic potential. ``Don't get your fingers too close, they sometimes bite.''
  
  Let $H \coloneqq \frac{1}{2}\of*{-\nthderiv{x}{2}+x^2-1}$. Then $\comm H {\ft\quad}=0$. We multiplied by $\frac{1}{2}$ so that it looks even more like a harmonic potential. Why did we add $-1$? Define
  \begin{align*}
    A &\coloneqq \frac{1}{\sqrt 2}\of*{\fstderiv{x} + x},\\
    A\adj &\coloneqq \frac{1}{\sqrt 2}\of*{-\fstderiv{x} + x}.
  \end{align*}
  We then have $H = A\adj A$. If $x$ and $y$ commute, then we have $\of*{x-y}\of*{x+y}=x^2-y^2$, but here $x$ and $\fstderiv{x}$ don't quite commute, hence the $-1$.

\emph{Here ends the lecture of 2013-03-06.}
\end{document}
"
% to automatically set PedanticNotation to true.
\newif\ifPedanticNotation
\ifdefined\targetPedantic
  \PedanticNotationtrue
\else
  \PedanticNotationfalse
\fi

%%% Constants.
% Constants are typeset in roman as per ISO 80000-2:2009
% \Pi is the circle constant.
\ifUtopia \renewcommand{\Pi}{{\mathrm \gp}} \fi
\ifComputerModern \renewcommand{\Pi}{\uppi} \fi
% \I is the maginary unit.
\newcommand\I{\mathrm{i}}
% \E is th base of the natural logarithm.
\newcommand\E{\mathrm{e}}

%%% Operators, standard functions.
%% Prefix operators.
% \diffd is the total differential.
\DeclareMathOperator{\diffd}{d}
% \FT is the prefix notation for the fourier transform.
\DeclareMathOperator{\FT}{\mathcal{F}}
% \End\of{X} is the endomorphism monoid of X.
\DeclareMathOperator{\End}{End}
% \Aut\of{X} is the endomorphism group of X.
\DeclareMathOperator{\Aut}{End}
% \GL\of{n,F} is the general linear group of degree n over F.
% \GL\of{V} is the general linear group of the vector space V.
% Use \GL\of{n} when the field is implicit.
\DeclareMathOperator{\GL}{GL}
% \SL\of{n,F} is the special linear group of degree n over F.
% Remarks from \GL apply.
\DeclareMathOperator{\SL}{SL}
% \Orth\of{n,F} is the orthogonal group of degree n over F.
% Remarks from \GL apply.
\DeclareMathOperator{\Orth}{O}
% \SO\of{n,F} is the special orthogonal group of degree n over F.
% Remarks from \GL apply.
\DeclareMathOperator{\SO}{SO}
%% Diacritics (one-parameter LaTeX commands).
% \conj z is the complex conjugate of z.
% By ISO 80000-2:2009 \overline{z} is the standard notation
% in mathematics, z^* the standard notation in physics.
\newcommand{\conj}[1]{\overline{z}}
% \ft is the compact notation for the fourier transform.
\newcommand\ft\widehat
% \rft is the compact notation for the reverse fourier transform.
\newcommand\rft\widecheck
%% Postfix operators.
% A\adj is the adjoint of A.
\newcommand{\adj}{^\dagger}
%% Many-argument operators.
% \deriv[n] x f is the nth derivative of f with respect to x.
% Use \deriv x f for the first derivative, and 
% \derivop[n] x for the operator 'nth derivative with respect to x.
\newcommand\deriv[3][]{\frac{\diffd^{#1} {#3}}{{\diffd {#2}}^{#1}}}
\newcommand{\derivop}[2][]{\deriv[#1]{#2}{}}
% \pderiv[n] {x_j} f is the nth derivative of f with respect to x_j.
% Remarks from \deriv apply.
\newcommand\pderiv[3][]{\frac{\partial^{#1} #3}{{\partial {#2}}^{#1}}}
\newcommand{\pderivop}[2][]{\pderiv[#1]{#2}{}}
% \pderiv\vx\miga is the composition of the partial derivation
% operators \pderiv[\ga_j]x_j{}.
\newcommand{\pd}[2]{{\partial_#1}^{#2}}

%%% Sets.
% \N is the set of natural numbers. 0\in\N by ISO 80000-2:2009.
\newcommand{\N}{\mathbb{N}}
% \Nstar is the set of stricly positive natural numbers.
\newcommand{\Nstar}{\mathbb{N}^*}
% \Z is the set of integers.
\newcommand{\Z}{\mathbb{Z}}
% \Q is the set of rationals.
\newcommand{\Q}{\mathbb{Q}}
% \R is the set of real numbers.
\newcommand{\R}{\mathbb{R}}
% \C is the set of complex numbers.
\newcommand{\C}{\mathbb{C}}
% \SchwartzSpace\of{\R^d} is the Schwartz space of \R^d.
\newcommand{\SchwartzSpace}{\mathcal{S}}

%%% Delimiters.
% \of is the bracketing for function arguments, as in
% f\of\vx, \diffd{x^2+1}. It is distinct from the normal parenthesising.
\newcommand\of[1]{\parentheses*{#1}}
% \pa is the parenthesising for associativity, as in \pa{x+y}^2.
\newcommand\pa[1]{\parentheses*{#1}}
% \tuple{a,b,c} is the tuple containing a, b and c in that order.
\newcommand\tuple[1]{\parentheses*{#1}}
% \set{a,b,c} is the set containing a, b and c.
\newcommand\set[1]{\curlyBrackets*{#1}}
%% Normlike delimiters.
% \abs is the absolute value on C or R, \abs z = \sqrt{z\conj{z}}.
\newcommand\abs[1]{\singleBars*{#1}}
% \norm is the Euclidean norm on R^n, \norm \vx = \sum_i \abs{x_i}^2.
\newcommand\norm[1]{\singleBars*{#1}}
% \Lnorm[p] is the Lp function norm, \Lnorm[p] f = \pa{Int_\Rn \abs{f}^p \diffd x}^\frac{1}{p}
% \Lnorm is the L\infty function norm,
% \Lnorm f = \setst{C\geq0}{\abs{f}\leq C \text{almost everywhere}}.
\newcommand\Lnorm[2][\infty]{\doubleBars*{#2}_{#1}}
% \SchwartzNorm\miga\migb{f} is defined as \sup_\{\vx\in\R^n}\abs{\vx^\miga\pd\vx\miga f\of\vx}.
% The Schwartz space is the space of functions for which it is finite for all \miga and \migb.
\newcommand{\SchwartzNorm}[3]{\doubleBars*{#3}_{#1,#2}}
% \total is the total of multi-indices, \total \miga = \sum_i \ga_i.
\newcommand\total[1]{\singleBars*{#1}}

%% Binary delimiters.
% Intervals.
% \intopen a b is the open interval from a excluded to b excluded.
% \intopcl a b is the left half-open interval from a excluded to b included.
% \intclop a b is the right half-open interval from a included to b excluded.
% \intclos a b is the closed interval from a included to b included.
% While this is not the main notation given by ISO 80000-2:2009,
% it is listed as valid alternative.
\newcommand{\intopen}[2]{\rsquareCommaLsquare{#1}{#2}}
\newcommand{\intopcl}[2]{\rsquareCommaRsquare{#1}{#2}}
\newcommand{\intclop}[2]{\lsquareCommaLsquare{#1}{#2}}
\newcommand{\intclos}[2]{\lsquareCommaRsquare{#1}{#2}}
% \commutator is the commutator of operators, \commutator A B = AB - BA.
\newcommand\commutator[2]{\lsquareCommaRsquare*{#1}{#2}}
% \scal is the standard scalar product on R^n, \scal \vx \vy = \sum_i x_i y_i
% we use \emptyDotEmpty as per ISO 80000-2:2009.
\newcommand\scal[2]{\emptyDotEmpty*{#1}{#2}}
% \pascal is used for expressions which need to be parenthesised if the notation for the
% scalar product \scal does not include delimiters, e.g. \emptyDotEmpty, but should not be if 
% it does, e.g. \angleCommaAngle. 'pascal' stands for pa[rentheses] scal[ar].
% Set \NeedPascal accordingly.
\newif\ifNeedPascal \NeedPascaltrue
\ifNeedPascal \newcommand{\pascal}[1]{\pa{#1}}
\else \newcommand{\pascal}[1]{#1} \fi
% \setst{x\in X}{P\of x} is the set of x in X such that P\of x is true.
\newcommand{\setst}[2]{\lcurlyBarRcurly{#1}{#2}}

%%% Variable names.
%% Greek letters.
% \g[*] refers to a variable named after a greek letter.
% The command for greek letterforms is of the form \g[one-letter Mathematica alias]
\newcommand\ga\alpha
\newcommand\gb\beta
\newcommand\gl\lambda
\newcommand\gj\varphi
\newcommand\gf\phi
\newcommand\gy\psi
\renewcommand\ge\epsilon
\newcommand\gd\delta
\newcommand\gx\xi
\newcommand\gh\eta
\newcommand\gp\pi
%% Vectors.
% Vectors are typeset in bold italic (isomath \vectorsym) as per ISO 80000-2:2009.
% We use the same convention for multi-indices.
% \v[symbol] is a vector, \mi[symbol] is a multi-index. \nullmi is the multi index (0,\ldots 0),
% \nullvec is the null vector.
\newcommand{\vx}{{\vectorsym x}}
\newcommand{\vy}{{\vectorsym y}}
\newcommand{\vk}{{\vectorsym k}}
\newcommand{\miga}{{\vectorsym \alpha}}
\newcommand{\migb}{{\vectorsym \beta}}
\newcommand{\nullmi}{{\vectorsym 0}}
\newcommand{\nullvec}{{\vectorsym 0}}

%%% Miscellaneous.
% \placeholder is a placeholder for a parameter,
% as in 'if f is linear, f\of{\gl\placeholder}=\gl f.'
\newcommand\placeholder{\,\:\cdot\:\,}

%%% Pedantic variants.
\ifPedanticNotation
  \renewcommand\of[1]{\squareBrackets*{#1}}
  \renewcommand\total[1]{\singleBars*{#1}_{\mathrm{MI}}}
  \renewcommand\norm[1]{\singleBars*{#1}_{2}}
  \renewcommand\abs[1]{\singleBars*{#1}_{\mathrm{abs}}}
  \renewcommand\tuple[1]{\parentheses*{#1}_{\mathrm{tuple}}}
  \renewcommand\commutator[2]{\lsquareCommaRsquare*{#1}{#2}_{\mathrm{commutator}}}
  \renewcommand\Lnorm[2][\infty]{\doubleBars*{#2}_{\mathcal{L}_{#1}}}
  \renewcommand{\intopen}[2]{\rsquareCommaLsquare{#1}{#2}_{\mathrm{interval}}}
  \renewcommand{\intopcl}[2]{\rsquareCommaRsquare{#1}{#2}_{\mathrm{interval}}}
  \renewcommand{\intclop}[2]{\lsquareCommaLsquare{#1}{#2}_{\mathrm{interval}}}
  \renewcommand{\intclos}[2]{\lsquareCommaRsquare{#1}{#2}_{\mathrm{interval}}}
\fi

%%%% Shorthands.

%% Several shorthands to reduce the number of '{', '}', '_', and other tiresome characters.
% Integral and summation over a space indicated below.
\newcommand\Int[1]{\int\limits_{#1}}
\newcommand{\Sum}[1]{\sum\limits_{#1}}

%% Frequently-used sets.
\newcommand{\Rn}{{\R^n}}
\newcommand{\Schwartz}{{\SchwartzSpace\of{\Rn}}}

%% Freqently-used expressions.
\newcommand{\sqftnrm}{\frac{1}{\pa{2\Pi}^n} }
\newcommand{\ftnrm}{\frac{1}{\pa{2\Pi}^\frac{n}{2}} }

%%%% Miscellaneous

% For comments in align environments
\newcommand\commutatorentbox[1]{\parbox{7.5cm}{#1}}


%%%% Title.

\title{Notes from the Methods of Mathematical Physics II courses of 2013-02-19{\slash}05-30 by Prof.~Dr.~Eugene~Trubowitz}
\author{Robin~Leroy and David~Nadlinger}


\begin{document}
  \maketitle
  \section{General properties of the Fourier transform}
  \noindent \emph{Here begins the lecture of 2013-03-06.}
  \begin{lemma}[``Your life in Fourier land depends on it.'']
    \begin{align}
      \label{FTLinearity}
      \begin{split}      
      \ft{\gj + \gy} &= \ft{\gj} + \ft{\gy},\\
      \ft{\gl \gj} &= \gl \ft{\gj} 
      \end{split}
      \\
      \label{FTUpperBound}
      \abs{\ft{\gj}(\vk)} &\leq \ftnrm\Lnorm[1]{\gj}<\infty, &&
      \Lnorm[1]{\gj} \coloneqq \Int\Rn \abs{\gj(\vx)} \diffd \vx
      \\
      \label{FTScaling}
      \ft{\gj\of{\gl\placeholder}}\of\vk &= \frac{1}{\abs{\gl}^n}\ft{\gj}\of{\frac\vk\gl}, &&
      \gl\neq 0
      \\
      \label{FTTranslation}
      \ft{T_\vy\gj}\of\vx &= \E^{\I\scal\vk\vy}\ft\gj\of\vk, &&
      \pa{T_\vy\gj}\of\vx \coloneqq \gj\of{\vx+\vy}
      \\
      \label{FTDerivative}
      \begin{split}
      \pderiv{x_j}\gj\of\vk &= \I k_j \ft\gj\of\vk \\ 
      \ft{x_j \gj}\of\vk &= \I \pderiv{k_j}{\ft\gj} \of\vk
      \end{split}
      \\
      \label{FTUnitary}
      \Int\Rn\ft\gj\of\vy \gy\of\vy \diffd\vy &= \Int\Rn \gj\of\vy \ft\gy\of\vy \diffd\vy
    \end{align}
    \begin{proof}
      Proof of (\ref{FTUpperBound}):
      \begin{equation*}
       \abs{\ft\gj\of\vk} 
        = \abs{\ftnrm\Int\Rn \gj\of\vx \E^{-\I \scal\vk\vx}\diffd\vx} 
        \leq \ftnrm\Int\Rn \abs{\gj\of\vx 
        \E^{-\I \scal\vk\vx}}\diffd\vx  = \ftnrm
        \underbrace{\Int\Rn \abs{\gj\of\vx}}_{\Lnorm[1]{\gj}}
        \underbrace{\abs{\E^{-\I\scal\vk\vx}}}_{1}\diffd\vx \\
      \end{equation*}
      Proof of existence of the integral $\Lnorm[1]\gj$:
      \begin{align*}
        \Int\Rn\abs{\gj\of\vx}\diffd\vx 
        &= \Int\Rn\pa{1+\norm{\vx}^2}^\frac{s}{2}\abs{\gj\of\vx}
        \frac{\diffd\vx}{\pa{1+\norm{\vx}^2}^\frac{s}{2}}  \\
        &\leq \SchwartzNorm s\nullmi\gj \Int\Rn\frac{1}{\pa{1+\norm{\vx}^2}^\frac{s}{2}}\diffd\vx\\
        &= \SchwartzNorm s\nullmi\gj \intop_0^\infty\frac{r^{n-1}}{\pa{1+r^2}^\frac{s}{2}}\diffd r
        < \infty &&\text{for } s\geq n+1\text{.}
      \end{align*}
      ``(\ref{FTLinearity}) I will not do!''
      Proof of (\ref{FTScaling}):
      \begin{align*}
        \ft{\gj\of{\gl\placeholder}} 
        &= \ftnrm\Int\Rn\gj\of{\gl\vx}\E^{-\I\scal\vk\vx} \diffd\vx \\
        &= \ftnrm\Int\Rn\gj\of\vy\E^{-\I\scal{\frac1\gl\vk}{\vy}} \diffd\of{\frac{\vy}{\gl}}
          && \text{where } \vx=\frac1\gl\vy\text{.} \\
        &= \frac{1}{\abs\gl^n} \underbrace{
          \ftnrm\Int\Rn\gj\of\vy\E^{-\I\scal{\frac\vk\gl}\vy}
          \diffd\vy}_{\ft\gj\of{\frac\vk{\gl}}}
      \end{align*}
      Proof of the second part of (\ref{FTDerivative}):
      \begin{align*}
        \ft{x_j\gj}\of\vx 
        &= \ftnrm\Int\Rn x_j\gj\of{\vx}\E^{-\I\scal\vk\vx}\diffd\vx \\
        &= \ftnrm
          \Int\Rn\gj\of{\vx}\I\pderivop{k_j}\E^{-\I\scal\vk\vx} \diffd\vx 
          && \text{as }\pderivop{k_j}\E^{-\I\scal\vk\vx} =
            \pderivop{k_j}\E^{-\I\sum_{r=0}^n k_j x_j}=-\I x_j \E^{-\I\scal\vk\vx}\\
        &= \I\pderivop{k_j}\ftnrm
          \Int\Rn\gj\of\vx\E^{-\I\scal\vk\vx} \diffd\vx \\
        &= \I\pderiv{k_j}{\ft\gj}\of\vk
      \end{align*}
      
      \emph{The professor starts whistling some elevator music while waiting for a student to write down the proof before he can wipe the board. When the student is done, the professor notices that there still is some room left on the blackboard and starts writing the rest there, without erasing anything.}
      
      Proof of \ref{FTUnitary}:
      \begin{align*}
        \Int\Rn\ft\gj\of\vy \gy\of\vy \diffd\vy
        &= \Int\Rn\ftnrm\Int\Rn\gj\of\vx
          \E^{-\I\scal\vy\vx}\diffd\vx\:\gy\of\vy\diffd\vy \\
        &= \Int\Rn\gj\of\vx\ftnrm\Int\Rn
          \E^{-\I\scal\vx\vy}\gy\of\vy\diffd\vy\diffd\vx
          && \commutatorentbox{As $\int_{\R^{2n}}\abs{\gj\of\vx
          \E^{-\I\scal\vy\vx}\gy\of\vy}\diffd\vx\diffd\vy<\infty$,
          we can use Fubini's theorem.} \\
        &=\Int\Rn\gj\of\vy \ft\gy\of\vy \diffd\vy
      \end{align*}
      
      
    \end{proof}
  \end{lemma}
  
  \begin{proposition}
  The Fourier transform $\ft\placeholder$ is a continuous bijective linear map from the Schwartz space $\Schwartz$ to itself. $\rft{\ft{\gj}}=\gj$, where $\rft\gy\of\vx\coloneqq\ftnrm \int_\Rn\gy\of\vk\E^{\I\scal\vk\vx}\diffd\vx$, read ``unhat'', ``bird'', ``seagull'', or whatever you like.
  
    \begin{proof}
      \begin{align*}
        \gj\in\Schwartz&\Rightarrow\vx^\miga\pd\vx\migb\in\Schwartz \\
        \ft{\pa{\vx^\miga\pd\vx\migb\gj}}\of\vk
        &=\I^{\total\miga}\pd\vk\miga\ft{\pa{\pd\vx\migb\gj}}\of\vk 
          && \text{from the second equality in (\ref{FTDerivative}), applied }\abs\miga\text{ times.} \\
        &=\I^{\total\miga}\pd\vk\miga\of{\I^{\total\migb}\vk^\migb\ft\gj}\of\vk \\
        &=\I^{\total\miga+\total\migb}\pd\vk\miga\of{\vk^\migb\ft\gj}\of\vk
      \end{align*}
      ``I should have done it the other way around. [...] Let's try it the other way around.''
      We want:
      \begin{equation*}
        \forall\miga,\migb,\sup_{\vk\in\Rn}\abs{\vk^\miga\pd\vk\migb\ft\gj\of\vk}<\infty
      \end{equation*}
      As that implies $\ft\gj\in\Schwartz$.
      \begin{align*}
        \ft{\pa{\pd\vx\miga\of{\vk^\migb\gj}}}\of\vk
        &=\I^{\total\miga+\total\migb}\vk^\miga\pd\vk\migb\ft\gj\of\vk
        && \commutatorentbox{by applying the first part of (\ref{FTDerivative}) $\total\miga
        $ times and the second part $\total\migb$ times.}\\
        \forall\miga,\migb,\abs{\vk^\miga\pd\vk\migb\ft\gj\of\vk}
        &\leq\ftnrm\Lnorm[1]{\pd\vx\migb\of{\vx^\migb\gj}}<\infty
        && \text{from (\ref{FTUpperBound}).}
      \end{align*}
      We now know $\ft\gj\in\Schwartz$.
      Let us prove $\rft{\ft\gj}=\gj$. ``When you do the wrong thing I'm gonna scream loudly.''
      \begin{align*}
        \rft{\ft\gj}\of\vx&=\ftnrm\Int\Rn\ft\gj\of\vk\E^{\I\scal\vk\vx}\diffd\vk \\
        &=\sqftnrm\Int\Rn\Int\Rn\gj\of\vy\E^{-\I\scal\vk\vy}\diffd\vy\:\E^{\I\scal\vk\vx}\diffd\vk
      \end{align*}
      We don't interchange the integrals here because that would lead to ugly calculations. ``If pou paint the walls before you start building, it's not a good idea.''
      At this point, somebody suggests replacing $\sqftnrm\int_\Rn \E^{\I\scal\vk{\pascal{\vx-\vy}}}\diffd\vk$ by $\gd\of*{\vx-\vy}$. ``It's plausible! --- WHY? [...] Ah I said it, so it's plausible.'' However, we would need to do some nasty calculations in order to do this. ``How are we going to do it so fast that you don't get bored, and yet in enough detail that he's convinced? Be sneaky.''
      \begin{align*}
        \rft{\ft{\gj}}
        &=\ftnrm\Int\Rn\ft\gj\of\vk\E^{\I\scal\vk\vx}
        \underbrace{\lim_{\ge\downarrow 0}\E^{-\ge\frac{\norm\vk^2}{2}}}_{\mathclap{\substack{1
        \text{ written in}\\\text{some other way}}}}\diffd\vk\\
        &=\lim_{\ge\downarrow 0}\ftnrm\Int\Rn\ft\gj\of\vk\E^{\I\scal\vk\vx}
        \E^{-\ge\frac{\norm\vk^2}{2}}&&\commutatorentbox{``Don't worry.'' This is actually a one-liner 
        using Lebesgue's dominated convergence theorem.}
      \end{align*}
      \emph{The 10-minute break ends with the loud noise of a metallic pointing stick hitting the desk.}
      \begin{align*}
        \ftnrm\Int\Rn\ft\gj\of\vk\E^{\I\scal\vk\vx}\E^{-\ge\frac{\norm\vk^2}{2}}\diffd\vk
        &=\ftnrm\Int\Rn\ftnrm\Int\Rn\gj\of\vy\E^{-\I\scal\vy\vk}\diffd\vy\:
        \E^{\I\scal\vk\vx}\E^{-\ge\frac{\norm\vk^2}{2}}\diffd\vk\\
        &=\ftnrm\Int\Rn\gj\of\vy\ftnrm\Int\Rn\E^{\I\scal\vk{\pascal{\vx-\vy}}}
        \E^{-\ge\frac{\norm\vk^2}{2}}\diffd\vk\diffd\vy
      \end{align*}
      Recall: ``every path leads to Rome and every Gaussian is in $\Schwartz$.'' Also, the last foot of a dactylic hexameter is always a spondee. 
      \begin{equation*}
        \ftnrm\Int\Rn\E^{-\ge\frac{\norm\vk^2}{2}}\E^{\I\scal{\pascal{\vy-\vx}}\vk}\diffd\vk
        =\frac{\E^{-\frac{1}{2\ge}\norm{\vx-\vy}^2}}{\ge^\frac{n}{2}}
      \end{equation*}
      We therefore get:
      \begin{align*}
         \ftnrm\Int\Rn\ft\gj\of\vk\E^{\I\scal\vk\vx}\E^{-\ge\frac{\norm\vk^2}{2}}\diffd\vk
         &=\ftnrm\Int\Rn\gj\of\vy\ftnrm\Int\Rn\E^{\I\scal\vk{\pascal{\vx-\vy}}}
        \E^{-\ge\frac{\norm\vk^2}{2}}\diffd\vk\diffd\vy\\
        &=\ftnrm\Int\Rn\gj\of\vy\frac{\E^{-\frac{1}{2\ge}\norm{\vx-\vy}^2}}
        {\ge^\frac{n}{2}}\diffd\vy \\
        &=\ftnrm\Int\Rn\gj\of{\vy\sqrt\ge+\vx}\frac{\E^{-\frac{1}{2\ge}\norm{\vy\sqrt\ge}^2}}
        {\ge^\frac{n}{2}}\diffd\of{\vy\sqrt\ge} && \text{``Epsilons everywhere!''}\\    
        &=\ftnrm\Int\Rn\gj\of{\vx+\vy\sqrt\ge}\E^{-\frac{1}{2}\norm\vy^2}\diffd\vy
        && \text{``Now we can paint.''}\\
        &\overset{\ge\downarrow 0}{\rightarrow}
        \ftnrm\gj\of\vx\Int\Rn\E^{-\frac{1}{2}\norm\vy^2}\diffd\vy =\gj\of\vx.
      \end{align*}
    \end{proof}
  \end{proposition}
  
  \section{Eigenfunctions of the Fourier transform}
  ``The main ideas of many things are right here.''
  We seek to find the solutions $\tuple{\gj,\gl}$ of
  \begin{equation*}
    \ft\gj=\gl\gj .
  \end{equation*}
  We shall find all of them. ``Not one is going to get away.'' Actually, we have already got one:
  \begin{equation*}
    \ft{\E^{-\frac{1}{2}\norm\vx^2}}=\E^{-\frac{1}{2}\norm\vx^2},
  \end{equation*}
  so we know that 1 is an eigenvalue. How do we find the others? ``Here you actually have to have an idea.''
  
  \paragraph{Idea}
  Find an $H\in\End\of\Schwartz$ such that $\commutator H {\ft\placeholder} = 0$. Then they have the same eigenspaces, so we just need to find the eigenfunctions of $H$ (see \emph{Finite Dimensional Quantum Mechanics}).
  
  Let $n=1$.
  \begin{equation*}
    \left.
    \begin{aligned}
      \ft{\pa{x^2\gj}}\of k &= -\derivop[2]{k}\ft\gj\of k \\
      \ft{\pa{\derivop[2]{x}\gj}}\of k &= -k^2 \ft\gj\of k
    \end{aligned}
    \right\rbrace \qquad \text{From the lemma your life depends on.}
  \end{equation*}
  Subtracting the second equality from the first one above:
  \begin{equation*}
    \FT\of{\pa{-\derivop[2]{x}+x^2}\gj}\of k = \pa{-\derivop[2]{k}+k^2}\FT\gj\of k,
  \end{equation*}
  where $\FT = \ft\placeholder$ because things are getting a bit too big for the hat.
  We smell a harmonic potential. ``Don't get your fingers too close, they sometimes bite.''
  
  Let $H \coloneqq \frac{1}{2}\pa{-\derivop[2]{x}+x^2-1}$. Then $\commutator H {\ft\placeholder}=0$. We multiplied by $\frac{1}{2}$ so that it looks even more like a harmonic potential (it is the Hamiltonian of the quantum harmonic oscillator). Why did we add $-1$? Define
  \begin{align*}
    A &\coloneqq \frac{1}{\sqrt 2}\pa{\derivop{x} + x},\\
    A\adj &\coloneqq \frac{1}{\sqrt 2}\pa{-\derivop{x} + x}.
  \end{align*}
  We then have $H = A\adj A$. If $a$ and $b$ commute, then we have $\pa{a-b}\pa{a+b}  = a^2-b^2$, but here $x$ and $\derivop{x}$ don't quite commute, hence the $-1$.

\emph{Here ends the lecture of 2013-03-06.}
\end{document}
